% INTRODUÇÃO-------------------------------------------------------------------

\chapter{INTRODUÇÃO}
\label{chap:introducao}

Apesar de parecer recente, o termo \textit fake  \textit news, ou notícia falsa, em português, é mais antigo do que aparenta. Segundo o dicionário Merriam-Webster, essa expressão é usada desde o final do século XIX. O termo é em inglês, mas se tornou popular em todo o mundo para denominar informações falsas que são publicadas, principalmente, em redes sociais.

Não é de hoje que mentiras são divulgadas como verdades, mas foi com o advento das redes sociais que esse tipo de publicação popularizou-se. A imprensa internacional começou a usar com mais frequência o termo \textit fake \textit news durante a eleição de 2016 nos Estados Unidos, na qual Donald Trump tornou-se presidente. Fake news é um termo em inglês e é usado para referir-se a falsas informações divulgadas, principalmente, em redes sociais.

Na época em que Trump foi eleito, algumas empresas especializadas identificaram uma série de sites com conteúdo duvidoso. A maioria das notícias divulgadas por esses sites explorava conteúdos sensacionalistas, envolvendo, em alguns casos, personalidades importantes, como a adversária de Trump, Hillary Clinton. \cite{fakenews}

Com as pessoas cada vez mais conectadas por meio de redes, criar e propagar notícias falsas acabaram se tornando a maneira mais eficaz de se fazer uma campanha política. O perigo é quando uma pessoa recebe uma mensagem e, sem checar sua veracidade, passa adiante uma informação que pode ser falsa. 

A ascensão das chamadas notícias falsas tem como objeto de grande preocupação em todo o mundo colocou no centro da discussão o papel de redes sociais como Facebook, Google, YouTube, Twitter e WhatsApp. Se por um lado é reconhecido que o fenômeno da desinformação é antigo, por outro lado é consenso entre pesquisadores, autoridades e empresas que a diferença no cenário atual de divulgação de conteúdos falsos está no alcance e na velocidade permitidos pelo compartilhamento de mensagens nesses ambientes.

Para tentar diminuir os questionamentos e o dano à imagem, diversas redes sociais vêm anunciando medidas para tentar combater a circulação das notícias falsas.

As redes sociais são terreno fértil para a difusão de notícias falsas por diferentes motivos. Alguns criadores desses conteúdos buscam divulgar uma ideia ou atacar uma pessoa, partido ou instituição. Outros têm motivação econômica, uma vez que a grande circulação de uma publicação gera interações, o que pode se traduzir em dinheiro a partir da lógica de veiculação de anúncios nessas plataformas. Foi o caso, por exemplo, de jovens da Macedônia que criaram perfis para difundir notícias falsas nas eleições dos Estados Unidos em 2016 como fonte de renda.
